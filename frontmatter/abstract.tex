
% Abstracts
% ------------------------------------------------------------------
% Include an abstract in the language that the thesis is written in,
% and if your native language is Finnish or Swedish, one in that language.

% Abstract in English
% ------------------------------------------------------------------
\thesisabstract{english}{The daily meeting is a commonplace agile software development method from the Scrum methodology. Agile methods are described as adaptable, but as they are often products of trial-and-error, successful adaptation can be difficult. Lean software development supports continuous improvement and offers principles based on which suitable methods can be created ad hoc. The introduction of lean calls into question the use of methods which add processual overhead, such as the daily.

This thesis studies the use of a daily in support of software development activities. An understanding is developed on how continuous improvement in practices may be analysed through the lens of practice-based activity theory. Guidelines leveraging the perspective of activity theory are offered for the development of practices where continuous improvement is a goal.

The theoretical objective is to find a means of description for the mechanism of continuous improvement in the practice of software development. Agile and lean methodology literature are joined by a perspective on improvement as a social process of innovation from activity theory, whereby improvement can be identified as transformations in a shared object of activity.

This thesis presents an ethnographical case study conducted within a software development team using a lean-inspired virtual kanban workflow model in their daily meetings. The team is observed over 20 dailies and audio recordings combined with screenshots of the kanban model are used to analyze how social interactions in the daily support continuous improvement.

The results of the study suggest that a daily practice may contribute to continuous improvement by providing a structure in which a visualization of work such as kanban can be used collaboratively, allowing failures in the practice to be brought up as disturbances which can be used as motives for transforming the practice. By pointing out disturbances as necessary for social innovation and showing the role they play in continuous improvement efforts, this thesis outlines a method for finding bottlenecks to learning in lean software development.}

% Abstract in Finnish
% ------------------------------------------------------------------
\thesisabstract{finnish}{yololoo}

% Reselect languages after the previous command leaks it
\selectlanguage{english}
