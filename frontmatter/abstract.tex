
% Abstracts
% ------------------------------------------------------------------
% Include an abstract in the language that the thesis is written in,
% and if your native language is Finnish or Swedish, one in that language.

% Abstract in English
% ------------------------------------------------------------------
\thesisabstract{english}{The daily meeting is a commonplace agile software development method from the Scrum methodology. Agile methods are described as adaptable, but as they are often products of trial-and-error, doing so successfully can be difficult. Lean software development offers principles based on which suitable methods can be created ad hoc and supports continuous improvement of ways of doing. The introduction of lean, however, calls into question the use of methods which add processual overhead, such as the daily.

This thesis studies the use of a daily in support of software development activities. In this thesis an understanding is developed on how continuous improvement in practices may be analysed through the lens of practice-based activity theory. The thesis offers guidelines leveraging the perspective of activity theory for the development of practices where continuous improvement is a goal.

The theoretical objective of this thesis is to find a means of description for the mechanism of continuous improvement in the practice of software development. Agile and lean methodology literature are joined by a perspective on improvement as a social process of innovation from activity theory, whereby improvement can be identified as transformations in a shared object of activity.

This thesis presents an ethnographical case study conducted within a software development team using a lean-inspired virtual kanban workflow model in their daily meetings. The team is observed over 20 dailies and audio recordings combined with screenshots of the kanban model are used to analyze how social interactions in the daily support continuous improvement.

The study suggests that a daily practice contributes to continuous improvement by providing a structure in which a visualization of work like kanban may be used collaboratively, allowing failures in the practice to be brought up through disturbances which can be used as motives for transforming the practice. In having delineated necessary ingredients for social innovation and investigated their applicability to observing continuous improvement, this thesis provides a basis for finding bottlenecks to learning through practice in software development.}

% Abstract in Finnish
% ------------------------------------------------------------------
\thesisabstract{finnish}{yololoo}

% Reselect languages after the previous command leaks it
\selectlanguage{english}
